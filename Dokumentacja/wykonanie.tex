\section{Praktyczna realizacja zadania}
Obraz z kamery zostaje przekazany do klasy obserwatora, zostaje tam przetworzony, oraz następuje wychwycenie twarzy na zdjęciu. Interfejs graficzny otrzymuje przechwyconą klatkę z kamery, wraz z twarzami obramowanymi prostokątami i wyświetla obraz na ekranie. W tym samym czasie z klatki zostają wyodrębione wszystkie twarze, które są jednocześnie skalowane do rozmiaru 100\begin{math}\times\end{math}100 pikseli, a ich barwy są zmieniane do skali szarości. Dzięki wykorzystaniu wzorca projektowego obserwator, przy każdej zmianie obrazu na kamerze klasa otrzymuje nowy obraz, a natychmiast po przetworzeniu obrazu klasy nasłuchujące otrzymują wymagane dane (klatka z obramowaniem, wycięte twarze).


\pagebreak[4]
