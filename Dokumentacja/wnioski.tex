\section{Wnioski i uwagi}
Niestety metoda używana przez nas nie jest zbyt dokładna i często obiekty są rozpoznawalne mylnie bądź też nie są rozpoznawane wcale chociaż powinny. Przechylenie głowy wprowadza już duże przekłamania. Lepszą metodą było by rozpoznawanie na podstawie odległości punktów charakterystycznych, niestety ograniczenie czasowe nie pozwoliło na implementację algorytmu. Dodatkowym utrudnieniem była mała ilość dokumentacji dla biblioteki w języku java, należało używać dokumentacji stworzonej dla C w której występowały różnice, oraz nie wszystkie funkcje otrzymały swój port dla języka java. Nie mniej jednak potężne możliwości biblioteki OpenCV pozwoliły na implementację skomplikowanych operacji w stosunkowo krótkim czasie. W naszym projekcie wykorzystaliśmy tylko małą część możliwości jakie oferowała biblioteka. Możliwe z jej pomocą jest m.in. śledzenie przedmiotów, rozpoznawanie kształtów i obiektów. Jednym z ciekawszych projektów opartych na tej bibliotece jest odstraszacz wiewiórek które wykradają jedzenie ptakom. Z pomocą biblioteki wykrywana jest wiewiórka, następnie działko wodne zbudowane z użyciem Arduino strzela w gryzonia. Na uwagę zasługuje fakt że biblioteka musi odróżnić ptaka od wiewiórki które często mają podobną barwę.
