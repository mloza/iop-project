
\section{Załączniki}
\subsection{Instalacja biblioteki OpcenCV w systemie Linux}
\begin{center}
\HRule \\[0.4cm]
\textsc{\Large Opis instalacji biblioteki openCV w systemie Linux\\ (dystrybucja Ubuntu 12.04 LTS)}
\HRule \\[0.4cm]
\end{center}

\begin{itemize}
\item Początkowo musimy pobrać bibliotekę \textbf{openCV} ze strony projektu 
\[
(\href{http://sourceforge.net/projects/opencvlibrary/}{http://sourceforge.net/projects/opencvlibrary/})
\]
 Pobieramy jej najnowszą wersję (my pracowaliśmy na wersji 2.4.2.).\\

\item Kolejnym krokiem jest zainstalowanie wszystkich bibliotek w naszym systemie, które są niezbędne do działania openCV. Zainstalujemy je przy pomocy polecenia \textbf{apt-get}. \\  

\begin{tabular}{p{15cm}}
 \textit{
apt-get install build-essential libjpeg-turbo8-dev libjpeg8-dev libjpeg-dev libgtk2.0-dev libavcodec-dev libavformat-dev libtiff4-dev cmake libpng++-dev libpng3
libpnglite-dev
zlib1g-dbg
freeglut3 libjasper-dev libjasper-runtime
libtiff-tools pngtools
libjpeg-turbo8-dbg libjpeg8-dbg
ffmpeg libav-tools libavdevice53 libavfilter2
libgstreamer0.10-0-dbg libgstreamer0.10-dev libxml2-dev
libslang2-dev libxine-dev libxine1-bin libxine1-ffmpeg
libunicap2 libunicap2-dev
libdc1394-22-dev libdc1394-utils libraw1394-dev
swig swig2.0
 libv4l-dev
libblas3gf libgfortran3 liblapack3gf python-numpy
libgstreamer-plugins-base0.10-dev}\\

\end{tabular} \\ 



\item Kolejnym krokiem jest wypakowanie biblioteki openCV. Robimy to przy pomocy polecenia: 

\begin{center}
\textit{tar -xvf OpenCV-2.4.2.tar.bz2}
\end{center}

\item Następnie wchodzimy do folderu z OpenCV

\begin{center}
\textit{cd OpenCV-2.4.2/}
\end{center}

\item i wykonujemy polecenie:

\begin{center}\textit{cmake -{}D CMAKE\_BUILD\_TYPE=RELEASE -D ENABLE\_SSE=OFF -D ENABLE\_SSE2=OFF -D ENABLE\_SSE3=OFF .} \end{center}

\item Następnie:
\begin{center}\textit{make}\end{center}
\item Oraz:
\begin{center}\textit{sudo make install}\end{center} \pagebreak[4]

\item Następnie musimy stworzyć w katalogu \textbf{\(/etc/ld.so.conf.d/\)} plik \textbf{opencv.conf}, w którym dokonamy wpisu:

\begin{center}\textit{/usr/local/lib}\end{center}

\item Po tych krokach konfigurujemy naszą bibliotekę poleceniem:

\begin{center}\textit{sudo ldconfig}\end{center}

\item Dodajemy na końcu pliku \(/etc/bash.bashrc\) lub \(\{katalog\_użytkownika\}/.bashrc\) wpis:
\begin{center}\mbox{export PKG\_CONFIG\_PATH=\$PKG\_CONFIG\_PATH:/usr/local/lib/pkgconfig} \end{center}

i restartujemy nasz komputer. Biblioteka openCV jest gotowa do użycia w naszym systemie.

\item Aby jej używać w Javie, musimy jeszcze do naszego projektu dołączyć \textbf{wrapper JavaCV}. Jest on dostępny na stronie
\[
 \href{http://code.google.com/p/javacv/downloads/list}{http://code.google.com/p/javacv/downloads/list}.
\]
 Po pobraniu archiwum \(javacv-0.2-bin.zip\) rozpakowujemy je. W eclipse, aby dołączyć zewnętrzny plik *.jar (mając otwarty projekt, do którego go dodajemy) wchodzimy kolejno:
\begin{equation}
Project \Rightarrow
 Properties \Rightarrow
 Java Build Path \Rightarrow
 Libraries \Rightarrow
 Add External JARs...
\end{equation}

i wybieramy \textbf{javacv.jar} z rozpakowanego archiwum. Nasz projekt powinien się skompilować i uruchomić bez problemów.
\end{itemize}

