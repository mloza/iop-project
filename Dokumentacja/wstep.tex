\section{Wstęp teoretyczny}

\subsection{Język JAVA}
Java jest obiektowym językiem programowania, powstałym w roku 1995. Poprzez standardowe jak i rozszerzone biblioteki wkracza w różnorodne rejony zastosowań takie jak np. karty inteligentne i elektronika, systemy zarządzania bazami danych, obsługa multimediów, internet, grafika 3D, kryptografia, itd. Co więcej JAVA jest niespotykanie bezpiecznym środowiskiem i umożliwia w znaczny sposób kontrolę i sterowanie bezpieczeństwem. Zdecydowanie różni się od innych języków trzeciej generacji tym, że jest językiem interpretowanym a nie kompilowanym. Oznacza to, że powstały w wyniku kompilacji kod wynikowy nie jest programem jaki można niezależnie uruchomić lecz stanowi tzw. Beta-kod, który jest interpretowany przez Maszynę Wirtualną (JavaVM) pracującą w określonym środowisku. Ze względu na kod nie istotne jest na jakim sprzęcie będzie uruchamiana aplikacja. Ważna jest tylko Maszyna Wirtualna. Jest to niezwykle ciekawy pomysł umożliwiający odcięcie się od wszystkich poziomów sprzętowo-programowych będących poniżej Maszyny Wirtualnej. Koncepcja ta jest powszechna również w samym języku JAVA, dzięki czemu poprzez stworzenie abstrakcyjnych klas i metod podstawowe biblioteki Javy nie muszą być nieustannie rozbudowywane.

\subsection{Rozpoznawanie obrazu}
Elementami składowymi kompletnego rozpoznawania obrazów są trzy poziomy: niskiego, średniego i wysokiego poziomu. Przetwarzanie niskiego poziomu obejmuje odbiór obrazu, przetwarzanie wstępne, oraz poprawę jakości obrazu (jak np. eliminacja zakłóceń, zmiana kontrastu czy filtracja). Przetwarzanie średniego poziomu polega na na segmentacji i wydzielaniu obiektów obrazu. Wysoki poziom odpowiada za klasyfikację, rozpoznanie i interpretację analizowanej sceny.
\pagebreak[4]
