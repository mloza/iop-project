\section{Założenia projektu}
Celem projektu \textsc{IntelligentEye} jest wykonanie programu wykrywającego podejrzane zachowania na podstawie analizy obrazu z kamery wideo.

\subsection{Technika wykonania}
Projekt jest stworzony w języku JAVA. Do rozpoznawania obrazu używana jest biblioteka \textbf{OpenCV}, opakowana za pomocą \textbf{JavaCV}.


\subsection{Analiza obrazu}
Obraz otrzymywany z kamery komputerowej jest analizowany pod kątem przede wszystkim rozpoznawania twarzy. Następnie przeszukana jest baza danych zawierająca portrety osób podejrzanych. W momencie wychwycenia podobieństwa program informuje użytkownika o niebezpieczeństwie. W miarę możliwości program będzie rozpoznawał również wszelkiego rodzaju podejrzane pakunki i przedmioty a także oznaki wskazujące na nerwowość i zaniepokojenie obserwowanych osób.

